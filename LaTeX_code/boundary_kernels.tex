% TeX file "boundary_kernels"

% Research Module in Econometrics & Statistics 
% Prof. Dr. Liebl & Dr. Christopher Walsh
% Winter 2021/22, M.Sc. Economics, Bonn University
% Xingyu Tao, Xuan Li, Sven Jacobs


\section{Boundary kernels} \label{sec:boundary_kernels}

Before the boundary-adaptivity of local linear regression had been uncovered, the classical approach to compensate the boundary effects was to apply special boundary kernels for estimation in the boundary region.
The usage of such kernels provides an asymptotic correction for the boundary bias,
i.e.\ the order of the asymptotic bias will be $\bigO(h^2)$ near the boundary as in the interior.

To understand the definition of a boundary kernel, we recall that the $\bigO(h)$ boundary bias in Theorem~\ref{theorem_3} resulted
because the truncated kernel moment $\tilde{\kappa}_1(K) = \int_{0}^{\infty} u K(u) \diff u$ does not vanish.
The problem for an arbitrary boundary point can be stated in the boundary framework from Section~\ref{subsec:theoretical_comparison},
where $\supp(K) = [-1, 1]$, $\supp(f) = [0, 1]$ and the boundary region is $[0, h) \cup (1-h, 1]$.  
Then, for a left boundary point $\int_{-\rho}^{1} u K(u) \diff u \neq 0$ causes the worse order.
A boundary kernel is then an asymmetrically supported kernel that restores the original moment properties \parencite{Gasser_1979}. 
Thus, for $x = \rho h$ a left boundary kernel $K_{\text{l}}(u, \rho)$ satisfies for $\rho \in [0, 1]$
\begin{equation}
	\int_{-\rho}^{1} u^j K_{\text{l}}(u, \rho) \diff u = \begin{cases} 
		1 & j = 0 \\
		0 & j = 1 \\
		\neq 0 & j = 2 
	\end{cases} \,.
\end{equation}
Notice that the kernel shape changes with the relative location of $x$ to the boundary, expressed through the parameter $\rho$. 

\textcite[Table 1]{Müller_1991} derived a boundary kernel that can be seen as a continuation of the Epanechnikov kernel onto the boundary.
The left-sided kernel is given for $\rho \in [0, 1]$ by
\begin{align}
	K_{\text{E,\,l}}(u, \rho) = \begin{cases}
		6 (1-u) (\rho + u) \frac{1}{(1+\rho)^3} \left\{ 1 + 5\left( \frac{1-\rho}{1+\rho} \right)^2 - 10\frac{1-\rho}{(1+\rho)^2} u \right\} & \text{if } u \in [-\rho, 1] \\
		0 & \text{else}
	\end{cases} \,. 
\end{align}
The equivalent for the upper boundary is obtained by $K_{\text{E,\,r}}(u, \rho) = K_{\text{E,\,l}}(-u, \rho)$.
Figure~\ref{fig:epanechnikov_boundary} plots these kernels for selected values of $\rho$.
For the special case $\rho = 1$ the boundary kernel agrees with the interior kernel,
and estimation at the boundaries is included as $\rho = 0$.
The figure illustrates that the kernel shape differs the more (i.e.\ a larger portion of negative weights is assigned),
the closer the evaluation point $x$ is to the boundary.
The way how boundary kernels work in practice is the same as for the effective kernel of local linear regression at the boundary in Figure~\ref{fig:effective_kernel_boundary}.
In fact, theoretical results suggest that the local linear estimator implicitly induces a boundary kernel-type bias correction \parencite{Ruppert_1994}.
\begin{figure}
	\centering
	\begin{tikzpicture}
		\begin{axis}
			[
			grid, grid style={white}, 
			samples=1000, 
			domain=-1:1, 
			xlabel=$u$,
			ylabel={$K_{\text{E,\,l}}(u, \rho)$},
			legend entries={$\rho = 1$, $\rho = 0.5$, $\rho = 0$},
			legend style={draw=none, font=\footnotesize}, legend cell align={left}, legend pos={north west},
			width=0.45\textwidth
			]
			
			\addplot[black]
			expression {epa_right(-x, 1)};
			
			\addplot[black, dashed]
			expression {epa_right(-x, 0.5)};
			
			\addplot[black, dotted]
			expression {epa_right(-x, 0)};
		\end{axis}
	\end{tikzpicture}
	\qquad
	\begin{tikzpicture}
		\begin{axis}
			[
			grid, grid style={white}, 
			samples=1000, 
			domain=-1:1, 
			xlabel=$u$,
			ylabel={$K_{\text{E,\,r}}(u, \rho)$},
			legend entries={$\rho = 1$, $\rho = 0.5$, $\rho = 0$},
			legend style={draw=none, font=\footnotesize}, legend cell align={left},
			width=0.45\textwidth
			]
			
			\addplot[black]
			expression {epa_right(x, 1)};
			
			\addplot[black, dashed]
			expression {epa_right(x, 0.5)};
			
			\addplot[black, dotted]
			expression {epa_right(x, 0)};
		\end{axis}
	\end{tikzpicture}
	\caption{Left- and right-sided Epanechnikov boundary kernels for different values of $\rho$}
	\label{fig:epanechnikov_boundary}
\end{figure}

Applying the boundary-adjusted Nadaraya-Watson estimator with the Epanechnikov boundary kernels to the simulated data from Figure~\ref{fig:nw},
we find that the explicit adjustment corrects for a substantial amount of the original boundary bias (see Figure~\ref{fig:nw_boundary_adjusted}).
However, performance is slightly worse than the local linear fit.
Figure~\ref{fig:effective_kernel_nw_boundary} reveals why.
Since \textcite{Müller_1991} required continuity at the endpoints of the boundary kernels (\enquote{smooth optimum boundary kernels}),
the closest observations to $x = 0$ receive considerably less weight compared to the local linear estimator.
A class of boundary kernels with greater asymptotic efficiency at the cost of discontinuity is given in \textcite{Müller_1994}.

\begin{remark}
	Local bandwidth choice is now much more involved since for each $\rho$ a different bias-variance trade-off arises
	(e.g.\ \cite[Section 4]{Müller_1991}).
\end{remark}