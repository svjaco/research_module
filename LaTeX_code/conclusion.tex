% TeX file "conclusion"

% Research Module in Econometrics & Statistics 
% Prof. Dr. Liebl & Dr. Christopher Walsh
% Winter 2021/22, M.Sc. Economics, Bonn University
% Xingyu Tao, Xuan Li, Sven Jacobs


\section{Conclusion} \label{sec:conclusion}

The paper discussed boundary effects in kernel regression.
We have seen that the Nadaraya-Watson estimator in general suffers from poor boundary behavior.
This is visually disturbing,
but also manifests in a larger order of the bias and thus a slower rate of convergence compared to the interior.
We focused on two methods to overcome the boundary problem.
Local linear regression preserves the order of the interior and was shown, theoretically and graphically,
to correct the bias exactly to first order.
The second approach, applying special boundary kernels when estimating in boundary regions,
instead provides an asymptotic bias adjustment.

A Monte-Carlo simulation study showed that the local linear estimator performs better than the boundary-adjusted Nadaraya-Watson estimator,
particularly for small sample sizes and non-uniform designs.
Moreover, the study revealed the relative advantage of Nadaraya-Watson for boundary estimation when estimation variability is the dominant component,
i.e.\ when the effective sample size is small or the regression function is rather flat.
A real-data application of the methods to determine the size of a causal effect in the RDD illustrated the need for boundary correction in practice,
as boundary effects can substantially bias estimates and inference.

Overall, in econometrics boundaries are often of special interest and the potential severe bias of Nadaraya-Watson then requires an adjustment.
Boundary kernels achieve an asymptotic correction,
but are rather complicated and tedious to implement in practice.
Local linear regression, in contrast, is a simple and intuitive method which automatically adapts to estimation at boundaries.
Besides, it has the favorable design-adaptivity property.
Therefore, we recommend local linear regression.

We restricted our analysis to the Nadaraya-Watson and the local linear estimator as two special cases of the local polynomial estimator to obtain the key insights.
Higher-order polynomials can further reduce the bias,
but at the expense of increased variability.
In practical applications often the local linear estimator is chosen.
An interesting extension of our research is to investigate boundary effects in higher dimensions.
Especially because the boundary region is larger and boundary effects may occur over much of the domain.
In fact, the fraction of points close to the boundary increases to one as the dimension grows \parencite[200]{Hastie_2009}.
Another possible extension is the estimation of curves that depend on the regression function,
e.g.\ derivatives.