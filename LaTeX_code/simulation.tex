% TeX file "simulation"

% Research Module in Econometrics & Statistics 
% Prof. Dr. Liebl & Dr. Christopher Walsh
% Winter 2021/22, M.Sc. Economics, Bonn University
% Xingyu Tao, Xuan Li, Sven Jacobs


\section{Simulation} \label{sec:simulation}

Our analysis of the estimators and their behavior near boundaries so far has been mainly based on asymptotic theory.
Thus, the correction of boundary kernels is only approximate for finite sample sizes.
For local linear regression we have shown the exact first-order bias correction, but the variability remains to be investigated.
To assess the finite sample properties of the estimators, we conduct a Monte-Carlo simulation study.
We first describe the set-up of the simulation.
Afterwards we present and discuss the results. 

\subsection{Set-up}

The kernel we are using is the Epanechnikov kernel.
Two reasons support this choice.
First, the kernel is asymptotically optimal for the estimation in the interior (see Appendix~\hyperref[appendix_2]{2}).
Second, its compact support on $[-1, 1]$ leads to faster computations and well-defined boundary regions.
The optimal kernel for the boundary region is the Triangular kernel \parencite{Cheng_1997}.
However, the efficiency loss of using the Epanechnikov kernel instead is negligible \parencite[684]{Hansen_2022}.
For the boundary-adjusted Nadaraya-Watson estimator we apply the Epanechnikov boundary kernels from Section~\ref{sec:boundary_kernels}.
As apparent from our theoretical analysis, the choice of the smoothing parameter $h$ is much more important.
Practical bandwidth selection is a broad topic on its own with several proposed procedures including, for example,
plug-in techniques and rules of thumb.
A popular approach, we use for the application in Section~\ref{sec:application}, is based on cross-validation.
The interested reader can find a detailed discussion of bandwidth selection in \textcite[Sections 3, 5.8]{Wand_1995}.
For the simulation we directly employ the asymptotically optimal global bandwidths from Theorem~\ref{theorem_2}.
We do not consider local selection due to the additional noise, its complexity for boundary points and complicated boundary regions that may arise.

Table~\ref{tab:simulation_overview} presents the six cases chosen for the simulation study.
We consider three regression functions whose graphs are shown in Figure~\ref{fig:simulation_functions}.
The function $m_1$ was used for simulating the data in the previous sections.
Function $m_2$ is a linear function with a peak in the interior, representing a linear boundary.
Function $m_3$ is similar to $m_2$, but the peak is shifted to lie approximately at the left boundary zero, representing a structured boundary.
In general, the regressor $X$ is taken to be uniformly distributed on $[0, 1]$ and
the error to be homoscedastic with distribution $\epsilon \sim \mathcal{N}(0, 0.25^2)$.
To study the performance of the estimators in a lower and higher signal-to-noise environment,
we modify the error variance when estimating $m_1$.
Lastly, we also look at the case of a non-uniform design as expressed by the design density $f_{\ast}$.
Figure~\ref{fig:clustered_design_density} reveals that $f_{\ast}$ reflects a clustered design, where fewer observations occur closer to the boundaries.

\renewcommand{\arraystretch}{1.5}	
\begin{table} \small
	\centering
	\captionabove{Selected cases for the simulation study}
	\label{tab:simulation_overview}
	\begin{tabular}{l l c l}  
		\toprule
		Regression function $m$ & Design & $\sigma^2(X)$ & Feature \\
		\midrule
		$m_1(x) = \sin(2 \pi x)$ 							  				 & $X \sim \mathcal{U}(0, 1)$ & $0.25^2$ & Example from Figure~\ref{fig:nw} \\
		$m_1(x) = \sin(2 \pi x)$ 							   				 & $X \sim \mathcal{U}(0, 1)$ & $0.10^2$ & High signal-to-noise ratio \\
		$m_1(x) = \sin(2 \pi x)$ 							   				 & $X \sim \mathcal{U}(0, 1)$ & $0.50^2$ & Low signal-to-noise ratio \\
		$m_1(x) = \sin(2 \pi x)$ 							                 & $X \sim f_{\ast}$ 		  & $0.25^2$ & Clustered (non-uniform) design \\
		$m_2(x) = 2 -2x + 2 \exp \left\{ \frac{-(x - 0.5)^2}{0.01} \right\}$ & $X \sim \mathcal{U}(0, 1)$ & $0.25^2$ & Linear function, peak in interior \\
		$m_3(x) = 2 -2x + 2 \exp \left\{ \frac{-(x - 0)^2}{0.01} \right\}$   & $X \sim \mathcal{U}(0, 1)$ & $0.25^2$ & Linear function, peak at boundary \\           
		\bottomrule \addlinespace[1ex]
		\multicolumn{4}{l}{$f_{\ast}(x) = 3 \cdot \{ 5/12 - (x - 0.5)^2 \}_{[0, \, 1]}$} 
	\end{tabular}	
\end{table}
\renewcommand{\arraystretch}{1.0}

The goodness-of-fit measure to evaluate the boundary behavior is the mean integrated squared error (MISE) over the left boundary region of the Nadaraya-Watson estimator,
i.e.\ $[0, h_{\text{opt}}^{\NW}(n))$.
For each of the six simulation cases we compute for seven representative sample sizes (from $n = 50$ to $n = 5000$) the percentage changes in the boundary MISE of the local linear estimator
and the boundary-adjusted Nadaraya-Watson estimator to the standard Nadaraya-Watson estimator.
Figure~\ref{fig:simulation_set-up} illustrates the main part of the procedure for the function $m_1$ and a sample size of $n = 100$.
The optimal bandwidth in this example is $h_{\text{opt}}^{\NW}(100) = 0.104$, giving the yellow-shaded boundary region.
The dark yellow area then reflects the integrated error of the unadjusted Nadaraya-Watson fit over the region of interest for a single random sample.
To obtain the MISE, we proceed as follows.
For $B$ Monte-Carlo repetitions with $n$ random observations,
we use numerical integration to calculate the integrated squared error (ISE).
Then, the expectation is approximated according to the law of large numbers by taking the average of the $B$ many ISE realizations:
\begin{equation}
	\MISE(n; \hat{m}) \approx \frac{1}{B} \sum_{j = 1}^{B} \ISE_j(n; \hat{m}) \,.
\end{equation}
The number of Monte-Carlo iterations equals $B = 10000$ for the first four sample sizes including $n = 500$ and $B = 1000$ otherwise.

The simulation study was conducted in $\mathsf{R}$. \nocite{R_2021}
All methods are self-implemented.

\subsection{Results and discussion}

We begin with the first case from Table~\ref{tab:simulation_overview}.
The simulation results are displayed in Table~\ref{tab:simulation_results_m_1}a.
We discuss this baseline case in greater detail since it reveals certain general patterns.
We note that for uniform designs (here on [0, 1]) the bandwidths for the Nadaraya-Watson (NW) and the local linear (LL) estimator coincide,
because then the asymptotic bias is the same (design bias is absent).
We see that the bandwidth gets smaller yet the effective sample size grows (from $nh = 5.95$ to $nh = 235$).
For $n = 50$ the left boundary region is about one tenth of the observation interval.
For this sample size the NW estimator performs best at the left boundary.
The MISE of the LL estimator is 789\% larger, for the boundary-adjusted NW estimator it is even 1578\%.
However, one should keep in mind that the absolute numbers are quite small.
From the sample size $n = 100$, LL regression yields a lower MISE, and the improvement increases with the sample size.
For $n = 5000$ the reduction amounts to 89\%.
In contrast, the application of boundary kernels pays off starting from a sample size of $n = 1000$.
Thereafter, the percentage reduction approaches the numbers achieved by LL regression.
For $n = 5000$ the reduction in the MISE amounts to 85\%.
Besides, we notice one unusual result for the adjusted NW estimator, namely a jump in the MISE change for 250 observations.
In summary, the simulated values show three developments.
First, LL regression quickly leads to improved boundary behavior which grows steadily.
Second, for smaller sample sizes the explicit boundary adjustment performs worse but achieves a lower MISE for larger samples.
And third, the larger the sample size the smaller is the gap between the two boundary correction methods.    

\begin{table}
	\centering
	\captionabove{Simulated percentage changes in the boundary MISE.
				  The regression function is $m_1$ (Subtables (a)--(d) correspond to cases one to four from Table~\ref{tab:simulation_overview}).}
	\label{tab:simulation_results_m_1}
	\begin{tabular}{r r r r}
		\toprule
		\multirow{2}[1]{*}{$n$} & \multirow{2}[1]{*}{Boundary region: $[0, h_{\text{opt}}^{\NW}(n))$} & \multicolumn{2}{c}{Change in MISE to NW [\%]} \\
		\cmidrule(lr){3-4} 
		& & Local linear & NW boundary-adjusted \\
		\midrule \addlinespace[2ex]
		\multicolumn{4}{c}{(a) $X \sim \mathcal{U}(0, 1)$, $\sigma^2(X) = 0.25^2$} \\[1ex]
		50   & [0, 0.119) & 	 789.29 & 	  1578.07 \\
		100  & [0, 0.104) & \redm 41.90 & 	  1143.28 \\
		250  & [0, 0.086) & \redm 65.13 & 	  1982.67 \\
		500  & [0, 0.075) & \redm 72.98 &   	14.92 \\
		1000 & [0, 0.065) & \redm 77.75 & \redm 65.84 \\
		2500 & [0, 0.055) & \redm 84.48 & \redm 77.33 \\
		5000 & [0, 0.047) & \redm 89.10 & \redm 84.82 \\[1ex]
		\multicolumn{4}{c}{(b) $X \sim \mathcal{U}(0, 1)$, $\sigma^2(X) = 0.10^2$} \\[1ex]
		50   & [0, 0.083) & 	1108.76 & 	  2782.59 \\
		100  & [0, 0.072) &  	 829.54 & 	  2651.04 \\
		250  & [0, 0.060) & \redm 82.21 & 	   660.80 \\
		500  & [0, 0.052) & \redm 86.45 & 	  1404.17 \\
		1000 & [0, 0.045) & \redm 88.91 & \redm 71.03 \\
		2500 & [0, 0.038) & \redm 92.47 & \redm 85.65 \\
		5000 & [0, 0.033) & \redm 94.70 & \redm 90.46 \\[1ex]
		\multicolumn{4}{c}{(c) $X \sim \mathcal{U}(0, 1)$, $\sigma^2(X) = 0.50^2$} \\[1ex]
		50   & [0, 0.157) & 	  29.02 & 	  1211.54 \\
		100  & [0, 0.137) & \redm 21.12 & 	  1713.05 \\
		250  & [0, 0.114) & \redm 43.81 & 	   160.39 \\
		500  & [0, 0.099) & \redm 55.65 & \redm 19.68 \\
		1000 & [0, 0.086) & \redm 63.34 & \redm 51.37 \\
		2500 & [0, 0.072) & \redm 73.52 & \redm 64.01 \\
		5000 & [0, 0.063) & \redm 81.22 & \redm 76.20 \\[1ex]
		\multicolumn{4}{c}{(d) $X \sim f_{\ast}$, $\sigma^2(X) = 0.25^2$} \\[1ex]
		50   & [0, 0.114) & 	1710.42 & 	   768.81 \\
		100  & [0, 0.099) &    78059.16 & 	  5265.00 \\
		250  & [0, 0.082) & \redm 47.65 & 	  1194.28 \\
		500  & [0, 0.072) & \redm 59.66 & 	   569.77 \\
		1000 & [0, 0.062) & \redm 66.92 & 	   169.63 \\
		2500 & [0, 0.052) & \redm 75.70 & \redm 43.91 \\
		5000 & [0, 0.045) & \redm 81.78 & \redm 69.66 \\
		\bottomrule
	\end{tabular}	
\end{table}

The reason for the better boundary performance of the LL estimator for all sample sizes except $n = 50$ is its first-order unbiasedness.
Since over the boundary regions the function $m_1$ can be well approximated by a linear function (see Figure~\ref{fig:ll_boundary}),
LL regression accomplishes a substantial bias reduction compared to the NW estimator.
However, the procedure is more variable than Nadaraya-Watson.
For $n = 50$ we only have on average six (noisy) observations in the boundary region.
Then, fitting locally a regression line is unstable and the variability dominates.
In fact, if we decrease the error variance while keeping the bandwidth fixed, at some point the bias reduction results in a smaller error for $n = 50$.
It can also be shown mathematically that the asymptotic variance of the LL estimator tends to become inflated near the boundary \parencite[Section 5.5]{Wand_1995}.
The results for the boundary-adjusted NW estimator illustrate the fact that the boundary correction is only asymptotically valid (as discussed in Section~\ref{sec:boundary_kernels}).
A sufficient amount of data ($n = 1000$) is necessary until the bias modification works in the considered simulation case.
The more data is effectively available, the less will the results between the LL estimator and the adjusted NW estimator differ
because we know from the theoretical treatment that both possess $\bigO(h^2)$ boundary bias.
Asymptotically the methods perform equivalently.
Lastly, to investigate the MISE increase for $n = 250$ when using boundary kernels,
we modified the noise level while everything else remained unchanged.
Only for a very small error variance close to the degenerate case, we observed a monotonically decreasing error.
Further investigation is necessary to understand this finding, also because the same behavior occurs for other set-ups. 

Next, starting from the baseline case we alter the variance of the homoscedastic error.
Table~\ref{tab:simulation_results_m_1}b reports the results for a smaller variance of $\sigma^2 = 0.1^2$ and Table~\ref{tab:simulation_results_m_1}c for $\sigma^2 = 0.5^2$, a variance four times the initial value $0.25^2$.
We might infer from the discussion of the baseline case that, for instance,
LL regression will achieve better (worse) results for a higher (lower) signal-to-noise ratio.
However, a change in the (conditional) error variance leads to a new bias-variance trade-off, and thus to different optimal bandwidths.
According to Theorem~\ref{theorem_2} a decrease (increase) of $\sigma^2$ results in smaller (larger) bandwidths.
Indeed, looking at Table~\ref{tab:simulation_results_m_1}b we notice a larger MISE over the boundary region for the LL estimator also for $n = 100$.
But for a sample size of $n = 250$ and beyond, the MISE reduction is stronger than in Table~\ref{tab:simulation_results_m_1}a
(e.g.\ 17 percentage points for $n = 250$).
A similar behavior is obtained for the boundary-adjusted NW estimator.

The reason for the increased MISE of both approaches relative to the NW estimator for smaller samples is the reduced effective sample size.
Even though observations falling into the boundary region are now on average more informative,
the lower number causing a higher estimation variability gives a relative advantage to NW.
If more data is effectively available, we see the dominating bias reduction effect.
Overall, Table~\ref{tab:simulation_results_m_1}b shows the same patterns as before (for the same reasons as before).
LL regression exhibits for all sample sizes better boundary behavior than the adjusted NW estimator,
the latter requires more data to achieve a MISE reduction, but the difference between both methods vanishes asymptotically.

For the increased error variance $\sigma^2 = 0.5^2$ we obtain opposite results to the lowest variance case,
which is in line with the explanation just given.
Compared to the baseline case bandwidths are now 32\% larger.
The MISE of LL regression for $n = 50$ is only 29\% larger, and the explicit boundary adjustment is already advantageous for 500 observations.
But the magnitude of the reductions is smaller compared to the two higher signal-to-noise cases.
For example, the reduction of the LL estimator for $\sigma^2 = 0.5^2$ and $n = 5000$ is with 81\% about the same as for $\sigma^2 = 0.1^2$ and only $n = 250$.
Estimation near the boundary is based on more observations, but these are less reliable due to a four-time higher error variance.

An interesting question is how the finite sample properties of the estimators depend on the design,
i.e.\ how the observations of the regressor are distributed.
To analyze the dependence, we again consider the baseline case and modify the design from being uniform to being clustered.
Compared to the uniform design the clustered design density $f_{\ast}$ from Table~\ref{tab:simulation_overview} gradually shifts half of the probability mass to the center (see Figure~\ref{fig:clustered_design_density}).
Hence, in practice the closer a region is located to the boundaries, the fewer observations will occur.
We point out that the non-uniform design induces a different bandwidth for the NW and the LL estimator (the latter ones are 4\% larger).
The design switch has essentially no effect on the size of the boundary region.
When comparing the MISE changes to those for the uniform design,
LL regression and overall also adjusted NW perform inferior.
For $n = 100$ the LL estimator shows a strong increase in the MISE, yet from a sample size of $n = 250$ reductions are achieved.
For the adjusted NW estimator instead $n = 2500$ observations are needed to get an improvement.

The worse boundary behavior (and the jump at $n = 100$) for the clustered design appears due to fewer observations near the boundary compared to the uniformly distributed regressor.
Moreover, the results indicate the fact that LL regression is design-adaptive while the adjusted NW method is not.
Boundary kernels are adaptive to the boundary but not to non-uniform designs near the boundary.
See \textcite{Hastie_1993} for a graphical illustration.

After discussing the results for $m_1$, we now examine the case of a linear conditional expectation function with a bump in the interior.
Figure~\ref{fig:m_2} displays the graph of $m_2$ and the simulation results are given in Table~\ref{tab:simulation_results_m_2}.
First, we notice the considerably smaller bandwidths compared to before.
Even for the smallest sample size $n = 50$, the bandwidth is already $h = 0.066$.
Second, although over the boundary regions the function $m_2$ is exactly linear,
neither boundary correction method leads to improved estimation, also not for the largest sample size.

The small bandwidths result from the peak in the interior of the observation interval (avoidance of excessive smoothing bias).
Then, LL regression has relatively high MISE values despite having zero estimation bias.
The functional form of $m_2$ was chosen to illustrate two points.
The NW estimator has an advantage in contexts where estimation variance is the decisive component,
i.e.\ when the effective sample size is small or estimation bias is low because the regression function is rather flat.
And, when evaluating boundary behavior it is important to choose a suitable research design.
If the smoothing parameter is selected globally, the boundary analysis may be dominated by an interior feature like a peak in our example.
To focus on boundary estimation without interference, we could either truncate the interval [0, 1] to exclude the peak
or try to incorporate local bandwidth choice (e.g.\ \cite{Fan_Gijbels_1992}).    

\begin{table}
	\centering
	\captionabove{Simulated percentage changes in the boundary MISE.
				  The regression function is $m_2$ (case five from Table~\ref{tab:simulation_overview}).}
	\label{tab:simulation_results_m_2}
	\begin{tabular}{r r r r}
		\toprule
		\multirow{2}[1]{*}{$n$} & \multirow{2}[1]{*}{Boundary region: $[0, h_{\text{opt}}^{\NW}(n))$} & \multicolumn{2}{c}{Change in MISE to NW [\%]} \\
		\cmidrule(lr){3-4}
		& & Local linear & NW boundary-adjusted \\
		\midrule
		50   & [0, 0.066) &   15953.93 &   989.86 \\
		100  & [0, 0.057) & 1040283.95 &   724.15 \\
		250  & [0, 0.048) & 	 86.23 & 17438.86 \\
		500  & [0, 0.042) & 	 65.39 &   293.78 \\
		1000 & [0, 0.036) & 	 54.40 &   114.32 \\
		2500 & [0, 0.030) & 	 42.89 &    74.65 \\
		5000 & [0, 0.026) & 	 33.27 &    66.51 \\
		\bottomrule
	\end{tabular}	
\end{table}

In the final simulation case the regression function $m_3$ is similar to $m_2$, but the peak is shifted to the left boundary (Figure~\ref{fig:m_3}).
Consequently, as can be seen from Table~\ref{tab:simulation_results_m_3} the boundary regions are somewhat larger.
Qualitatively the results are now again similar to the ones for the function $m_1$.
For the smallest (effective) sample sizes LL regression is worse, but then quickly improves on NW in a constant manner
(the lower reduction for $n = 1000$ is an artifact of the fewer Monte-Carlo repetitions).
Regarding the adjusted NW estimator it takes longer (until $n = 1000$) to pay off.
When the boundary region contains at least a few observations ($nh = 250 \cdot 0.055 \approx 14$) the bias reduction achieved by LL regression takes over.
Due to the strong negative slope of $m_3$ in the boundary regions, NW has substantial downward bias.
In contrast, by modeling slopes LL regression is able to capture this structured boundary.
The asymptotic adjustment of boundary kernels instead is visible not until 1000 observations.

\begin{table}
	\centering
	\captionabove{Simulated percentage changes in the boundary MISE.
				  The regression function is $m_3$ (case six from Table~\ref{tab:simulation_overview}).}
	\label{tab:simulation_results_m_3}
	\begin{tabular}{r r r r}
		\toprule
		\multirow{2}[1]{*}{$n$} & \multirow{2}[1]{*}{Boundary region: $[0, h_{\text{opt}}^{\NW}(n))$} & \multicolumn{2}{c}{Change in MISE to NW [\%]} \\
		\cmidrule(lr){3-4}
		& & Local linear & NW boundary-adjusted \\
		\midrule
		50   & [0, 0.076) & 	 999.26 & 	 76196.73 \\
		100  & [0, 0.066) &    41922.93 & 	  5420.96 \\
		250  & [0, 0.055) & \redm 63.17 & 	   356.63 \\
		500  & [0, 0.048) & \redm 65.57 & 	   343.59 \\
		1000 & [0, 0.042) & \redm 64.64 & \redm 39.99 \\
		2500 & [0, 0.035) & \redm 66.81 & \redm 56.64 \\
		5000 & [0, 0.030) & \redm 69.30 & \redm 63.45 \\
		\bottomrule
	\end{tabular}	
\end{table}

We briefly summarize the simulation findings.
The LL estimator performs better than the adjusted NW estimator, particularly for small sample sizes and non-uniform designs.
Only for samples as large as 5000 or more the methods are equivalent.
When effective sample sizes are small and also when the regression function near the boundary is rather flat (estimation bias is negligible) NW is favorable.
When there is some structure, i.e.\ curvature, near the boundary and the effective sample size is roughly not smaller than $nh = 15$, then the LL estimator is beneficial.